\documentclass{beamer}

\input{preamble.tex}



\begin{document}

\imageframe{./lecture_includes/cover_understanding.png}

\section{Where do (Good) Instruments Come From?}

\subsection{True Lotteries}
\begin{frame}{Subtlties of the Validity Condition}
\begin{itemize}
\item To apply IV, we need to make a good case for instrument validity \\ (note we can always check relevance!)\pause{}
\medskip
\item Consider our simple causal model, $Y_i=\alpha+\beta D_i+\varepsilon_i$. Validity $Cov(Z_i,\varepsilon_i)=0$ intuitively requires two distinct assumptions:\smallskip\pause{}
\begin{itemize}
\item \emph{As-good-as-random assignment}: individuals with higher/lower potential earnings face the same distribution of $Z_i$\smallskip
\item \emph{Exclusion}: the ``assignment'' of $Z_i$ only affects $Y_i$ through $D_i$
\end{itemize}
\medskip\pause{}

\item Confusingly, old-school econometrics texts sometimes refer to $Cov(Z_i,\varepsilon_i)=0$ as the ``exclusion restriction''
\smallskip\pause{}

\begin{itemize}
\item More modern IV texts take care to distinguish between these two conceptually distinct requirements... 
\end{itemize}
\end{itemize}
\end{frame}

\begin{frame}{A Valid Instrument}
\begin{center}
\includegraphics[scale=0.6]{./lecture_includes/dag3.png}
\end{center}
\end{frame}

\begin{frame}{A Violation of As-Good-As-Random Assignment}
\begin{center}
\includegraphics[scale=0.6]{./lecture_includes/dag4.png}
\end{center}
\end{frame}

\begin{frame}{A Violation of Exclusion}
\begin{center}
\includegraphics[scale=0.6]{./lecture_includes/dag5.png}
\end{center}
\end{frame}

\begin{frame}{Where do IVs Come From? 1) True Lotteries}
\begin{itemize}
\item One sure-fire way to ensure that a $Z_i$ is as-good-as-randomly assigned is... \pause{} to randomly assign it!\pause{}\smallskip
\begin{itemize}
\item Some of the best IVs come from lotteries, either run by the researcher (e.g. an RCT) or so-called ``natural experiments''
\smallskip
\item We still need to worry about violations of the exclusion restriction
\smallskip
\item Relevance holds when $Z_i$ has some effect on $X_i$
\end{itemize}\pause{}
\medskip

\item ``Gold standard'' IV: a randomized offer to participate in a program, with $X_i$ recording program participation\smallskip
\begin{itemize}
\item Exclusion restriction likely to hold for any $Y_i$, by construction
\smallskip
\item Relevance almost guaranteed (provided people want the program!)
\end{itemize}
\end{itemize}
\end{frame}

\begin{frame}{Example: Charter School Lotteries}
\begin{itemize}
\item Abdulkadiroglu et al. (2016) are interested in whether going to a ``charter'' middle school increases standardized test scores\smallskip
\begin{itemize}
\item Charter students tend to score better, but we worry about selection
\smallskip
\item History of doubting educational inputs, since Coleman (1966)
\end{itemize}
\medskip\pause{}
\item We leverage an institutional feature of charters: \emph{admission lotteries}\smallskip
\begin{itemize}
\item When more kids want to enroll than there are seats, admission offers $Z_i\in\{0,1\}$ are effectively drawn from a hat
\smallskip
\item Offers plausibly only affect later test scores $Y_i$ by changing charter enrollment $D_i\in\{0,1\}$, so are plausibly valid instruments
\smallskip
\item We need to control for lottery fixed effects (``risk sets'') to make $Z_i$ as-good-as-randomly assigned -- more on this soon
\end{itemize}
\medskip\pause{}
\item We study a particular charter (UP Academy), which is ``takeover''\smallskip
\begin{itemize}
\item Two offer IVs: ``immediate'' (on lottery night) and from a waitlist
\end{itemize}
\end{itemize}
\end{frame}

\begin{frame}{Lottery IV Estimates of UP Test Score Effects}

\begin{center}
\includegraphics[scale=0.32]{./lecture_includes/charters1.png}
\end{center}

\end{frame}

\subsection{Natural Experiments}
\begin{frame}{Where do IVs Come From? 2) Natural Experiments}
\begin{itemize}
\item Without appealing to literal randomization, we may credibly argue $Z_i$ is as-good-as-randomly assigned conditional on some $\mathbf{W}_i$\smallskip
\begin{itemize}
\item Such ``natural experiments'' rely on a selection-on-observables argument (for $Z_i$, instead $D_i$)
\smallskip
\item Still worry about exclusion: $Z_i$ cannot affect $Y_i$ except through $D_i$
\end{itemize}\pause{}\medskip
\item Angrist and Krueger (1991) famously estimate labor market returns to schooling with a creative IV: student quarter-of-birth\smallskip
\begin{itemize}
\item Compulsory schooling requirements prevent students from dropping before the day they turn 16 (used to be more binding)
\smallskip
\item Fixed school start dates mean students who drop out at 16 get more or less schooling depending on their birth date\pause{}
\smallskip
\item Quarter-of-birth seems quasi-randomly assigned --- is it excludable? See Buckles and Hungerman (2013)...
\end{itemize}
\end{itemize}
\end{frame}

\begin{frame}{The Quarter-of-Birth Natural Experiment: Visualized}

\vspace{-0.3cm}
\begin{center}
\includegraphics[scale=0.45]{./lecture_includes/qob1.png}
\end{center}

\end{frame}

\begin{frame}{Quarter-of-Birth IV Estimates of Returns to Schooling}

\begin{center}
\includegraphics[scale=0.32]{./lecture_includes/qob2.png}
\end{center}

\end{frame}

\subsection{Panel Data}
\begin{frame}{Where do IVs Come From? 3) Panel Data}
\begin{itemize}
\item We might also combine IV + difference-in-difference identification\smallskip
\begin{itemize}
\item E.g. instrument with $Z_i\times Post_t$, controlling for $Z_i$ and $Post_t$ FEs
\smallskip
\item This requires two parallel trends assumptions, for the RF and FS
\smallskip
\item Still need to worry about the exclusion restriction, as always
\end{itemize}\pause{}\medskip
\item Abdulkadiroglu et al. (2016) complement their lottery analysis of takeover charters with an instrumented diff-in-diff analysis\smallskip
\begin{itemize}
\item Students enrolled in the ``legacy'' public school were eligible for being ``grandfathered'' into UP, without having to apply to the charter
\smallskip
\item We compare their trends in test scores \& enrollment to a matched comparison group of observably-similar students at other schools
\end{itemize}
\end{itemize}
\end{frame}

\begin{frame}{Grandfathering IV: Visualized}

\vspace{-0.2cm}
\begin{center}
\includegraphics[scale=0.29]{./lecture_includes/charters2.png}
\end{center}

\end{frame}

\begin{frame}{Grandfathering IV Estimates of UP Test Score Effects}

\begin{center}
\includegraphics[scale=0.25]{./lecture_includes/charters3.png}
\end{center}

\end{frame}

\section{2SLS Mechanics}

\subsection{Just-Identified IV}
\begin{frame}{Just-Identified IV}
Stuff about just-identified IV
\end{frame}

\subsection{Overidentification}
\begin{frame}{Overidentification}
Stuff about overidentification
\end{frame}

\section{Weak and Many Instruments}

\subsection{Weak IV}
\begin{frame}{Weak IV}
Stuff about weak IV
\end{frame}

\subsection{Many IVs}
\begin{frame}{Many IVs}
Stuff about many IVs
\end{frame}

\end{document}
